\documentclass{amsart}
\usepackage{amssymb}
\usepackage[noadjust]{cite}
\usepackage{enumerate}
\usepackage{mathrsfs}
\usepackage{tikz}
\usepackage{tikz-cd}
\usepackage{wrapfig}
\usepackage{caption}
\usepackage[position=b]{subcaption}


\newtheorem{theorem}{Theorem}
\newtheorem{prop}[theorem]{Proposition}
\newtheorem{lemma}[theorem]{Lemma}
\newtheorem{cor}[theorem]{Corollary}
\theoremstyle{definition}
\newtheorem{definition}[theorem]{Definition}
\newtheorem{question}[theorem]{Question}

\DeclareMathOperator{\Coh}{Coh}
\DeclareMathOperator{\Ext}{Ext}
\DeclareMathOperator{\cExt}{\mathscr{E} \! \textit{xt}}
\DeclareMathOperator{\Gr}{Gr}
\DeclareMathOperator{\Hom}{Hom}
\DeclareMathOperator{\Ho}{H}
\newcommand{\cHom}{\mathcal{H} \textit{om}}
\DeclareMathOperator{\id}{id}
\DeclareMathOperator{\Ob}{Ob}
\DeclareMathOperator{\Sch}{{\bf Sch}}
\DeclareMathOperator{\Sing}{Sing}
\DeclareMathOperator{\Spec}{Spec}
\DeclareMathOperator{\Vect}{{\bf Vec}}

\renewcommand{\AA}{\mathcal{A}}
\newcommand{\BB}{\mathcal{B}}
\newcommand{\CC}{\mathbb{C}}
\newcommand{\CL}{\mathcal{C}}
\newcommand{\DD}{\mathcal{D}}
\newcommand{\FF}{\mathscr{F}}
\newcommand{\GG}{\mathscr{G}}
\newcommand{\HH}{\mathbb{H}}
\newcommand{\II}{\mathscr{I}}
\newcommand{\JJ}{\mathscr{J}}
\newcommand{\bL}{\textbf{L}}
\newcommand{\LL}{\mathcal{L}}
\newcommand{\OO}{\mathcal{O}}
\newcommand{\PP}{\mathbb{P}}
\newcommand{\qis}{\simeq_{qis}}
\newcommand{\qisf}{\simeq_{qis,filt}}
\newcommand{\QQ}{\mathbb{Q}}
\newcommand{\RR}{\mathbb{R}}
\newcommand{\bR}{\textbf{R}}
\newcommand{\ZZ}{\mathbb{Z}}
\newcommand{\otimesL}{\otimes^{\bf L}}

\newcommand*\xbar[1]{%
  \hbox{%
    \vbox{%
      \hrule height 0.5pt % The actual bar
      \kern0.5ex%         % Distance between bar and symbol
      \hbox{%
        \kern-0.1em%      % Shortening on the left side
        \ensuremath{#1}%
        \kern-0.1em%      % Shortening on the right side
      }%
    }%
  }%
} 

\newcommand{\tu}{\underline{2}}
\DeclareMathOperator{\red}{red}
\newcommand{\DB}{\underline{\Omega}}
\DeclareMathOperator{\tot}{tot}

\begin{document}
	
\begin{lemma}
	Let $I$ be a thin, finite category without loops and with an initial object. Then if $X$ is an $I$-scheme, there is a finite collection of $V_{i}$, affine open sub-$I$-schemes of $X$ with $\{V_{i\alpha}\}_i$ open covers of $X_\alpha$ for each $\alpha \in I$.
\end{lemma}

\begin{proof}
	First we note that the objects in $I$ can be ordered such that each object only maps to the objects of larger order, that is, $\Ob I$ has a bijection with $\{0, \dots, N\}$ such that if there is a morphism $i \rightarrow j$, then $i \leq j$. The initial object must be 0 in this construction.
	
	Thus we can construct $X$ inductively as follows: let $X^0 = X_0$, and let $X^1$ be the full subdiagram of $X$ with only $X_0$ and $X_1$, etc., so that $X^n$ is obtained from $X^{n-1}$ by taking the diagram containing $X^{n-1}$, $X_n$, and all the maps $f_{n\alpha}:X_n \rightarrow X_\alpha$ out of $X_n$ in the diagram $X$, by construction these all map to objects in $X^{n-1}$.
	
	Construct $V_i^n$ as follows: for $X^0$ choose an open affine cover of $X_0$, and for $X_n$, for each $V_i^{n-1}$ choose an open affine cover $U_{ij}$ of $\cap_\alpha f_{n\alpha}^{-1}(V_{i\alpha})$ (here we take an empty intersection to be the entire space). Let $V_{ij}^n$ be the diagram $U_{ij} \xrightarrow{\{f_{n\alpha}\}_\alpha} V_i$.
	
	We wish to prove that for each $n$, this construction forms a cover as desired for $X^n$. This would follow by the construction and the following:
	
	Claim: For each $n$, $\cap_\alpha f_{n\alpha}^{-1}(V_{i\alpha})$ covers $X_n$.
	
	We prove this by induction. For $X_0$, the intersection the entire space, so this is true. In general, suppose $p \in X_n$. Let $\alpha_1, \dots, \alpha_m$ be the objects in $I$ which $n$ maps to, in their order. We must show that there is an element of $V_i^{n-1}$ which has $f_{\alpha_j}(p)$ in each $V_{i \alpha_j}^{n-1}$.
	
	There is a $V_{i_1}^{\alpha_1}$ containing $f_{\alpha_1}(p)$, by construction. Inducting on the $\alpha_i$'s, assume now that there is a $V_{i_{k-1}}^{\alpha_{k-1}}$ containing each of the $\{f_{\alpha_1}(p), \dots, f_{\alpha_{k-1}}(p)\}$. Then by the hypotheses, if $g_\beta$'s are the maps out of $X_{\alpha_k}$, $\cap_\beta g_{\beta}^{-1}(V_{i_{k-1}\beta})$ is a cover of $X_{\alpha_k}$, thus by construction there is a $V_{i_{k-1}j}^{\alpha_k}$ including $\{f_{\alpha_1}(p), \dots, f_{\alpha_{k}}(p)\}$.
\end{proof}


\bibliography{library}{}
\bibliographystyle{plain}
\end{document}