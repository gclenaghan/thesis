\documentclass{amsart}
\usepackage{amssymb}
\usepackage[noadjust]{cite}
\usepackage{enumerate}
\usepackage{mathrsfs}
\usepackage{tikz}
\usepackage{tikz-cd}
\usepackage{wrapfig}
\usepackage{caption}
\usepackage[position=b]{subcaption}


\newtheorem{theorem}{Theorem}
\newtheorem{prop}[theorem]{Proposition}
\newtheorem{lemma}[theorem]{Lemma}
\newtheorem{cor}[theorem]{Corollary}
\theoremstyle{definition}
\newtheorem{definition}[theorem]{Definition}
\newtheorem{question}[theorem]{Question}

\DeclareMathOperator{\Coh}{Coh}
\DeclareMathOperator{\Ext}{Ext}
\DeclareMathOperator{\cExt}{\mathscr{E} \! \textit{xt}}
\DeclareMathOperator{\Gr}{Gr}
\DeclareMathOperator{\Hom}{Hom}
\DeclareMathOperator{\Ho}{H}
\newcommand{\cHom}{\mathcal{H} \textit{om}}
\DeclareMathOperator{\id}{id}
\DeclareMathOperator{\Ob}{Ob}
\DeclareMathOperator{\Sch}{{\bf Sch}}
\DeclareMathOperator{\Sing}{Sing}
\DeclareMathOperator{\Spec}{Spec}
\DeclareMathOperator{\Vect}{{\bf Vec}}

\renewcommand{\AA}{\mathcal{A}}
\newcommand{\BB}{\mathcal{B}}
\newcommand{\CC}{\mathbb{C}}
\newcommand{\CL}{\mathcal{C}}
\newcommand{\DD}{\mathcal{D}}
\newcommand{\EE}{\mathscr{E}}
\newcommand{\FF}{\mathscr{F}}
\newcommand{\GG}{\mathscr{G}}
\newcommand{\HH}{\mathbb{H}}
\newcommand{\II}{\mathscr{I}}
\newcommand{\JJ}{\mathscr{J}}
\newcommand{\bL}{\textbf{L}}
\newcommand{\LL}{\mathcal{L}}
\newcommand{\OO}{\mathcal{O}}
\newcommand{\PP}{\mathbb{P}}
\newcommand{\qis}{\simeq_{qis}}
\newcommand{\qisf}{\simeq_{qis,filt}}
\newcommand{\QQ}{\mathbb{Q}}
\newcommand{\RR}{\mathbb{R}}
\newcommand{\bR}{\textbf{R}}
\newcommand{\ZZ}{\mathbb{Z}}
\newcommand{\otimesL}{\otimes^{\bf L}}

\newcommand*\xbar[1]{%
  \hbox{%
    \vbox{%
      \hrule height 0.5pt % The actual bar
      \kern0.5ex%         % Distance between bar and symbol
      \hbox{%
        \kern-0.1em%      % Shortening on the left side
        \ensuremath{#1}%
        \kern-0.1em%      % Shortening on the right side
      }%
    }%
  }%
} 

\newcommand{\tu}{\underline{2}}
\DeclareMathOperator{\red}{red}
\newcommand{\DB}{\underline{\Omega}}
\DeclareMathOperator{\tot}{tot}

\begin{document}
	
\begin{lemma}
	\label{lem:affinecover}
	Let $I$ be a thin, finite category without loops and with an initial object.
	Then if $X$ is a quasi-compact $I$-scheme, there is a finite collection of $V_{i}$, affine open sub-$I$-schemes of $X$ with $\{V_{i\alpha}\}_i$ open covers of $X_\alpha$ for each $\alpha \in I$.
\end{lemma}

\begin{proof}
	First we note that the objects in $I$ can be ordered such that each object only maps to the objects of larger order, that is, $\Ob I$ has a bijection with $\{0, \dots, N\}$ such that if there is a morphism $i \rightarrow j$, then $i \leq j$.
	The initial object must be 0 in this construction.
	
	Thus we can construct $X$ inductively as follows: let $X^0 = X_0$, and let $X^1$ be the full subdiagram of $X$ with only $X_0$ and $X_1$, etc., so that $X^n$ is obtained from $X^{n-1}$ by taking the diagram containing $X^{n-1}$, $X_n$, and all the maps $f_{n\alpha}:X_n \rightarrow X_\alpha$ out of $X_n$ in the diagram $X$, by construction these all map to objects in $X^{n-1}$.
	
	Construct $V_i^n$ as follows: for $X^0$ choose an open affine cover of $X_0$, and for $X_n$, for each $V_i^{n-1}$ choose an open affine cover $U_{ij}$ of $\cap_\alpha f_{n\alpha}^{-1}(V_{i\alpha})$ (here we take an empty intersection to be the entire space).
	Let $V_{ij}^n$ be the diagram $U_{ij} \xrightarrow{\{f_{n\alpha}\}_\alpha} V_i$.
	
	We wish to prove that for each $n$, this construction forms a cover as desired for $X^n$.
	This would follow by the construction and the following:
	
	Claim: For each $n$, $\cap_\alpha f_{n\alpha}^{-1}(V_{i\alpha})$ covers $X_n$.
	
	We prove this by induction.
	For $X_0$, the intersection the entire space, so this is true.
	In general, suppose $p \in X_n$.
	Let $\alpha_1, \dots, \alpha_m$ be the objects in $I$ which $n$ maps to, in their order.
	We must show that there is an element of $V_i^{n-1}$ which has $f_{\alpha_j}(p)$ in each $V_{i \alpha_j}^{n-1}$.
	
	There is a $V_{i_1}^{\alpha_1}$ containing $f_{\alpha_1}(p)$, by construction.
	Inducting on the $\alpha_i$'s, assume now that there is a $V_{i_{k-1}}^{\alpha_{k-1}}$ containing each of the $\{f_{\alpha_1}(p), \dots, f_{\alpha_{k-1}}(p)\}$.
	Then by the hypotheses, if $g_\beta$'s are the maps out of $X_{\alpha_k}$, $\cap_\beta g_{\beta}^{-1}(V_{i_{k-1}\beta})$ is a cover of $X_{\alpha_k}$, thus by construction there is a $V_{i_{k-1}j}^{\alpha_k}$ including $\{f_{\alpha_1}(p), \dots, f_{\alpha_{k}}(p)\}$.
\end{proof}

\begin{definition}
	Let $X_\bullet$ be a diagram of schemes, $\FF_\bullet$ and $\GG_\bullet$ two sheaves over $X_\bullet$.
	Define $\cHom_{X_\bullet}(\FF_\bullet, \GG_\bullet)$ as follows: for each $\alpha \in I$, and $U \subseteq X_\alpha$ open, let
	\[
		(\cHom_{X_\bullet}(\FF_\bullet, \GG_\bullet))_\alpha(U) = \Hom_{X_\bullet/U} ( i_U^* \FF_\bullet , i_U^* \GG_\bullet).
	\]
	This gives a sheaf on each $X_\alpha$.
	For $X_\phi : X_\alpha \rightarrow X_\beta$, $U \subseteq X_\beta$, there is a map $X_\bullet / X_\phi^{-1}(U) \rightarrow X_\bullet / U$ such that the $i$ maps commute, and putting this together gives maps $(\cHom_{X_\bullet}(\FF_\bullet, \GG_\bullet))_\beta(U) \rightarrow (\cHom_{X_\bullet}(\FF_\bullet, \GG_\bullet))_\alpha(X_\phi^{-1}(U))$ as needed.
\end{definition}

\begin{prop}
	\label{prp:homqcaffine}
	If $X_\bullet$ has affine arrows, then for any $\FF_\bullet$ coherent, $\GG_\bullet$ (quasi-)coherent, $\cHom_{X_\bullet}(\FF_\bullet, \GG_\bullet)$ is (quasi-)coherent.
\end{prop}
\begin{proof}
	Enough to show that for each $X_\alpha$, that the sheaf $(\cHom_{X_\bullet}(\FF_\bullet, \GG_\bullet))_\alpha$ is (quasi-)coherent, which can be done locally, and furthermore for each $U \subseteq X_\alpha$, the definition of $\cHom$ only depends on the diagram $X_\bullet / U$ so we can restrict to proving this for a terminal scheme in a diagram, and furthermore restrict that this scheme is affine.
	Because $X_\bullet$ has affine arrows, the entire diagram must be affine.
	
	Now define
	\[
		\EE_{\alpha\beta} = \begin{cases}
								\OO_{X_\beta} & \text{if } \beta \in I/\alpha \\
									0 & \text{otherwise}
							\end{cases}
	\]
	with the maps between nonzero sheaves being the natural $\OO_{X_\beta} \rightarrow X_{\phi *} \OO_{X_\eta}$.
	
	We can directly compute 
	\[
		\cHom(\EE_{\alpha \bullet}, \GG_\bullet)_\beta = \begin{cases}
													\GG_\beta  & \text{if } \beta \in I/\alpha \\
													0 & \text{otherwise}
													\end{cases},
	\]
	so this is (quasi-)coherent by the same assumption on $\GG_\bullet$.
	
	Each $\FF_\alpha$ is coherent on an affine scheme, so is the cokernel of a map of finite rank free modules, these give sequences, surjective on the $\alpha$ component of the diagram:
	\[
		\EE_{\alpha \bullet}^{\oplus m_\alpha} \rightarrow \EE_{\alpha \bullet}^{\oplus n_\alpha} \rightarrow \FF_\bullet.
	\]
	Combining these gives a $\FF$ as a cokernel:
	\[
		\bigoplus_\alpha \EE_{\alpha \bullet}^{\oplus m_\alpha} \rightarrow \bigoplus_\alpha \EE_{\alpha \bullet}^{\oplus n_\alpha} \rightarrow \FF_\bullet \rightarrow 0.
	\]
	
	Applying $\cHom_{X_\bullet}(-,\GG_\bullet)$,
	\[
		0 \rightarrow \cHom_{X_\bullet}(\FF_\bullet, \GG_\bullet) \rightarrow \cHom_{X_\bullet}(\bigoplus \EE_{\alpha\bullet}, \GG_\bullet) \rightarrow \cHom_{X_\bullet}(\bigoplus \EE_{\alpha\bullet}, \GG_\bullet),
	\]
	and since the direct sums are finite this can be rewritten as
	\[
		0 \rightarrow \cHom_{X_\bullet}(\FF_\bullet, \GG_\bullet) \rightarrow \bigoplus \cHom_{X_\bullet}(\EE_{\alpha\bullet}, \GG_\bullet) \rightarrow \bigoplus \cHom_{X_\bullet}(\EE_{\alpha\bullet}, \GG_\bullet).
	\]
	Now since finite direct sums and kernels of (quasi-)coherent sheaves are (quasi-)coherent, we are done.
\end{proof}

\begin{theorem}
	If $X_\bullet$ is a concentrated $I$-scheme, then if $\FF_\bullet$ is coherent and $\GG_\bullet$ is (quasi-)coherent, $\cHom_{X_\bullet}(\FF_\bullet, \GG_\bullet)$ is (quasi-)coherent.
\end{theorem}
\begin{proof}
	As before, reduce to the case with $X_\bullet$ a diagram with an affine, terminal component $X_0$. Then apply lemma \ref{lem:affinecover} to $X_\bullet$ to have an affine diagram cover $\{V_{i\bullet}\}$. Each intersection $V_{i \bullet} \cap V_{j \bullet}$ is quasi-compact by quasi-separability, so we can reapply lemma \ref{lem:affinecover} to have affine diagram covers $\{V_{ijk\bullet}\}$.
	
	Have maps $\cHom_{X_\bullet}(\FF_\bullet, \GG_\bullet)_0 \rightarrow \cHom_{V_{i\bullet}}(\FF_\bullet |_{V_{i\bullet}}, \GG_\bullet |_{V_{i \bullet}})_0$ given by restriction, and in fact if we are given $\phi_i \in \cHom_{V_{i\bullet}}(\FF_\bullet |_{V_{i\bullet}}, \GG_\bullet |_{V_{i \bullet}})_0$ for each $i$ such that $\phi_i |_{V_{ijk \bullet}} = \phi_j |_{V_{ijk \bullet}}$ for each $i, j, k$, then the morphisms glue uniquely to give $\psi \in \cHom_{X_\bullet}(\FF_\bullet, \GG_\bullet)_0$.
	
	This establishes the equalizer diagram
	\[
	\begin{tikzcd}
		\cHom_{X_\bullet}(\FF_\bullet, \GG_\bullet)_0 \arrow{r} & \bigoplus_i \cHom_{V_{i\bullet}}(\FF_\bullet |_{V_{i\bullet}}, \GG_\bullet |_{V_{i \bullet}})_0 \arrow[yshift=.5ex]{r} \arrow[yshift=-.5ex]{r} & \bigoplus_{i,j,k} \cHom_{V_{ijk\bullet}}(\FF_\bullet |_{V_{ijk\bullet}}, \GG_\bullet |_{V_{ijk \bullet}})_0
	\end{tikzcd}
	\]
	by proposition \ref{prp:homqcaffine}, the right two terms are (quasi-)coherent, so as is the term on the left.
\end{proof}

\bibliography{library}{}
\bibliographystyle{plain}
\end{document}